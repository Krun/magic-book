\documentclass[a4paper,12pt]{article}
\usepackage[utf8]{inputenc}
\usepackage{mathtools}

% Allow the change of line spacing
\usepackage{setspace}
\usepackage{tabularx}
\usepackage{graphicx}
\usepackage[usenames,dvipsnames,table]{xcolor}

%\usepackage{hyperref}
%\usepackage{breakurl}

%opening
%\title{Trainmining}
%\author{Grupo de Sistemas Inteligentes \\ Universidad Politécnica de Madrid}


\begin{document}
\newcommand\litem[1]{\item{\bfseries #1 }}
\renewcommand{\arraystretch}{1.5} %Makes tables less crammed

\newcommand\headcell[1]{%
  \multicolumn{1}{c|}{\cellcolor{MidnightBlue}\bfseries\sffamily\textcolor{white}{#1}}
}

%\renewcommand{\abstractname}{Executive Summary}
%\begin{abstract}
%
%\end{abstract}

% Set line spacing to 1.5
\onehalfspacing

\include{entregable4_titlepage}
%\maketitle

\pagenumbering{roman}
\section*{Executive Summary}
\addcontentsline{toc}{section}{Executive Summary} % si queremos que aparezca en el índice
This document describes the design and implementation of a prototype predictive module for Thales' railway maintenance network. This implementation relies on already existing predictive rules which have been previously obtained from data mining procedures.

The prototype provides therefore a way to apply said obtained knowledge to actual situations, evaluating the rules and determining an output for each situation. It works as a rule-engine which takes the current situation as an input and outputs a list of predicted events along with an associated confidence for each of them. The system has been implemented in the form of a Java module, and can therefore be used as a standalone system or be integrated onto larger systems at convenience. It relies on the JBoss Drools Expert library, which provides an efficient and reliable rule-engine environment.

In this document, the architecture of the implemented prototype will be explained on detail. Furthermore, performance specifications are described as a result of several testing procedures.


\newpage
\tableofcontents % indice de contenidos
\addcontentsline{toc}{section}{Contents} % para que aparezca en el indice de 
\cleardoublepage
\addcontentsline{toc}{section}{List of Figures} % para que aparezca en el indice de contenidos
\listoffigures % indice de figuras

\cleardoublepage
\addcontentsline{toc}{section}{List of tables} % para que aparezca en el indice de contenidos
\listoftables % indice de tablas
\cleardoublepage

\setcounter{page}{1}
\pagenumbering{arabic}

\section{Module description}

In this section we will provide a general description of the implemented prototype. This prototype allows the usage of already existing association rules, which have been already obtained as the result of a \emph{Data Mining} procedure. These rules do not offer any functionality by themselves, as a system is needed to check whether their conditions are fulfilled and therefore a prediction can be made.

Rules are simply textual information in the form of "When A and B happen together, C has 80\% chances of happening". This information would be useful for an operator who might be manually checking events and would be able to expect C after seeing A and B. However, real systems usually have a much larger set of possible events, and many more events happening during each observation, and therefore an automated system is needed to perform these operations.

We call such a system a \emph{Rule engine}\cite{liang2009openrulebench}. A rule engine is simply a system which evaluates input conditions and fires the rules which comply with these conditions, outputting the result of said rules. In our case, the system will take as input a set of current events, and output a list of predictions along with their probabilities.

\subsection{Parameters and interfaces}
Depending on our needs for each situation, our system can obtain predictions for several periods of time, different types of systems and different input lengths. Specifically, we count on rules for four different stations of different characteristics (Albacete, Antequera, Segovia and Sevilla) and for three different time windows (one day, two days and seven days). Each of these prediction modes needs a different type of input and needs to use a different set of rules. This distinction has therefore been made at the time of generating the rules, and now those rules need to be loaded accordingly to the kind of prediction we want to obtain.

In other words, said parameters (station type and time window) fixes the set of rules to be used and the length of the input to be provided. This association also works in the other direction: by using an specific rule set and an specific input, we are already defining the execution parameters. Therefore, both the station type and prediction time window are irrelevant for the correct function of our predictive module. It is responsibility of the operator or executing system to select the appropriate rule set and provide an appropriate input. This allows new execution options to be added or updated at any time, without the need of modifying the module in any way. If we generate a new set of rules expecting a whole month of input and which will generate a month of predictions, we just need to load it and know what it is generating. Also, if we generate rules for a new type of maintenance station we just need to load them on the module and provide an input according to that new type of station.

\subsection{Architecture}


\bibliographystyle{plain} 
\bibliography{datamining}

\end{document}


