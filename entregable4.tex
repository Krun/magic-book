\documentclass[a4paper,12pt]{article}
\usepackage[utf8]{inputenc}
\usepackage{mathtools}

% Allow the change of line spacing
\usepackage{setspace}
\usepackage{tabularx}
\usepackage{graphicx}
\usepackage[usenames,dvipsnames,table]{xcolor}

%\usepackage{hyperref}
%\usepackage{breakurl}

%opening
%\title{Trainmining}
%\author{Grupo de Sistemas Inteligentes \\ Universidad Politécnica de Madrid}


\begin{document}
\newcommand\litem[1]{\item{\bfseries #1 }}
\renewcommand{\arraystretch}{1.5} %Makes tables less crammed

\newcommand\headcell[1]{%
  \multicolumn{1}{c|}{\cellcolor{MidnightBlue}\bfseries\sffamily\textcolor{white}{#1}}
}

%\renewcommand{\abstractname}{Executive Summary}
%\begin{abstract}
%
%\end{abstract}

% Set line spacing to 1.5
\onehalfspacing

\include{entregable4_titlepage}
%\maketitle

\pagenumbering{roman}
\section*{Executive Summary}
\addcontentsline{toc}{section}{Executive Summary} % si queremos que aparezca en el índice
This document describes the design and implementation of a prototype predictive module for Thales' railway maintenance network. This implementation relies on already existing predictive rules which have been previously obtained from data mining procedures.

The prototype provides therefore a way to apply said obtained knowledge to actual situations, evaluating the rules and determining an output for each situation. It works as a rule-engine which takes the current situation as an input and outputs a list of predicted events along with an associated confidence for each of them. The system has been implemented in the form of a Java module, and can therefore be used as a standalone system or be integrated onto larger systems at convenience. It relies on the JBoss Drools Expert library, which provides an efficient and reliable rule-engine environment.

In this document, the architecture of the implemented prototype will be explained on detail. Furthermore, performance specifications are described as a result of several testing procedures.


\newpage
\tableofcontents % indice de contenidos
\addcontentsline{toc}{section}{Contents} % para que aparezca en el indice de 
\cleardoublepage
\addcontentsline{toc}{section}{List of Figures} % para que aparezca en el indice de contenidos
\listoffigures % indice de figuras

\cleardoublepage
\addcontentsline{toc}{section}{List of tables} % para que aparezca en el indice de contenidos
\listoftables % indice de tablas
\cleardoublepage

\setcounter{page}{1}
\pagenumbering{arabic}
\section{Introduction}

\section{Architecture}




\bibliographystyle{plain} 
\bibliography{datamining}

\end{document}


