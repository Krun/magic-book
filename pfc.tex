\documentclass[a4paper,10pt]{book}
\usepackage[utf8x]{inputenc}

\begin{document}
\newcommand\litem[1]{\item{\bfseries #1 }}


\chapter{Introduction}
\section{Knowledge Discovery in Databases}
Knowledge Discovery in Databases - or \emph{KDD} - is a term used to describe the procedure of acquiring high-level knowledge from low-level data. As a formal definition Knowledge discovery in databases in the non- 

trivial process of identifying valid, novel, potentially useful, and ultimately understandable patterns in data.\cite{Fay
\begin{enumerate}
 \litem{Understanding the application domain.} It is important to understand the environment we are studying.
 \litem{Creating a target dataset.} Amongst all the available data, we will usually have to select a dataset or subset of variables to be used for data discovery
 \litem{Data cleaning and preprocessing.} Discerning which data is actually relevant and significant for our study, and which is merely noise or outliers which should be disregarded. Operations such as noise modelling or mapping of missing and unknown values are also taken in this step.
 \litem{Data reduction and transformation.} Reducing the effective number of variables under consideration and finding useful features to represent the data.
 \litem{Choosing data mining function.} Deciding the purpose of the model derived by the data mining algorithm.
 \litem{Choosing data mining algorithms.} Selecting methods to be used to search for patterns in the data.
 \litem{Data mining.} Application of chosen algorithms.
 \litem{Interpretation.} Consists on interpreting the discovered patterns, removing those redundant or irrelevant and translating the useful ones into understandable terms.
 \litem{Consolidation of discovered knowledge.} The discovered knowledge is finally consolidated in an appropriate form. Depending on the context of our project, it might be simply documented or integrated in predictive modules for the analysed systems.
\end{enumerate}





1. Developing an understanding of the application domain, the rel- evant prior knowledge, and the goals of the end user. 
2. Creating a target data set, selecting a data set, or focusing on a subset of variables or data samples, on which discovery  is to be performed. 
3. Data cleaning and preprocessing. 
4. Data reduction and transformation. 
5. Choosing the data-mining task. 
6. Choosing the data-mining algorithm(s). 
7. Data mining. 
8. Evaluating output of Step 7. 
9. Consolidating discovered knowledge: incorporating this knowledge into the performance system, or simply documenting it and reporting it to users.

Data Integration: First of all the data are collected and integrated from all the different sources.
Data Selection: We may not all the data we have collected in the first step. So in this step we select only those data which we think useful for data mining.
Data Cleaning: The data we have collected are not clean and may contain errors, missing values, noisy or inconsistent data. So we need to apply different techniques to get rid of such anomalies.
Data Transformation: The data even after cleaning are not ready for mining as we need to transform them into forms appropriate for mining. The techniques used to accomplish this are smoothing, aggregation, normalization etc.
Data Mining: Now we are ready to apply data mining techniques on the data to discover the interesting patterns. Techniques like clustering and association analysis are among the many different techniques used for data mining.
Pattern Evaluation and Knowledge Presentation: This step involves visualization, transformation, removing redundant patterns etc from the patterns we generated.
Decisions / Use of Discovered Knowledge: This step helps user to make use of the knowledge acquired to take better decisions.

\bibliographystyle{plain} 
\bibliography{datamining}

\end{document}
