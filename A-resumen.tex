\phantomsection
\chapter*{Resumen}
\addcontentsline{toc}{chapter}{Resumen}

%\begin{center}
%\textbf{\large Resumen}
%\end{center}

Esta memoria presenta el resultado del proyecto \emph{Trainmining}, realizado por el Grupo de Sistemas Inteligentes junto con Thales España. El proyecto consiste en el desarrollo e implementación de un sistema
de predicción de fallos para una red de mantenimiento ferroviario, mediante la aplicación de técnicas de minería de datos.

En concreto, el objetivo del proyecto es ser capaz de predecir eventos futuros en la red ferroviaria (correspondientes a alertas y fallos de diversos sistemas) basándonos en los eventos que hayan ocurrido hasta el momento. La motivación de esta tarea es ser capaces de predecir los fallos con antelación, de forma que se puedan prevenir o planificar su reparación de una forma más eficiente.

En el documento se presenta una visión general de los procesos de minería de datos, así como de algunos algoritmos considerados para la implementación del sistema. Mediante la utilización de estos algoritmos, podemos obtener conocimiento predictivo de bases de datos de eventos pasados, descubriendo de forma automática relaciones, patrones o secuencias que nos permitirán en el futuro predecir algunos eventos con una confianza determinada. Se describe asimismo los procedimientos pertinentes de validación, para evitar el fenómeno conocido como \emph{sobreentrenamiento} y mejorar la funcionalidad global del sistema. Tras este proceso de validación se procede a presentar un análisis y evaluación de los resultados.

Con el fin de demostrar el funcionamiento del sistema desarrollado y la utilidad del conocimiento adquirido, se ha desarrollado un prototipo que genera predicciones a partir de eventos de entrada.

Finalmente, se presentan algunas alternativas al trabajo realizado, así como conclusiones y posibles líneas de trabajo futuro.

\vfill
\textbf{Palabras clave:} Minería de datos, predicción, mantenimiento predictivo, eventos, secuencias, alarmas, aprendizaje.

