\phantomsection
\chapter*{Abstract}
\addcontentsline{toc}{chapter}{Abstract}

The \emph{Trainmining} project, developed by the \emph{Intelligent Systems Group} along with \emph{Thales Spain}, consists on the development and implementation of a failure prediction system for a railway maintenance network, as the result of a Data Mining process.

Specifically, the project aims to provide the ability to predict future events in the railway network (corresponding to alarms and failures of diverse systems) taking as a base the events which have happened in the close past. The purpose of this task is being able to predict failure events before they happen, in order to prevent them or plan ahead for their solution in the most efficient way.

We present a general overview of Data Mining processes, as well as of some algorithms we have considered for the implementation of the system. Through these algorihtms, we can obtain predictive knowlege from databases of past events, automatically discovering relations, patterns or sequences which could allow us to predict other events with a certain confidence. Evaluation and validation processes are as well defined and described, in order to avoid overfitting and improve the overall performance of the system. After this validation process, an analysis of results is presented.

In order to demonstrate the utility of the developed system and the usefulness of the acquired knowledge, a predictive module prototype has been implemented, which generates predictions based on given input data.

Finally, alternative methods for the performed work are presented, as well as conclusions and possible future development lines.

\vfill
\textbf{Keywords:} Data mining, prediction, predictive maintenance, events, alarms, sequences, machine learning.
