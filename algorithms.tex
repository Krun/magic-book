\section{Choosing an appropriate Data Mining algorithm}

\emph{Data Mining} comprises a large amount of different algorithms which can be used for the Knowledge Discovery process. Depending on the nature of the data on which we will apply these algorithms, and on the kind of knowledge we expect or want to acquire, we will need algorithms of different types.

Different algorithms can usually be classified in the following categories:

\begin{enumerate}
 \litem{Classification:} Learning a function that maps an item into predefined classes.
 \litem{Regression:} Learning a function that maps an item to a predicted variable.
 \litem{Segmentation:} Identifying a set of clusters to categorise the data.
 \litem{Summarization:} Finding a compact description for the data.
 \litem{Association:} Finding significant dependencies between different variables.\cite{Zhao2003}
 \litem{Sequence analysis:} Finding frequent sequences or episodes in data.\cite{Zhao2003a}\cite{Weiss2002}
\end{enumerate}

The most immediate type of technique to be applied is \emph{Sequence analysis}. When dealing with event-based problems, such as this one, it is very important to maintain the information given by the time variable. Alarms (and therefore their related events) are most likely to happen in sequential patterns. That is, the information about the order and timing of the alarms is very likely to be essential for the prediction tasks.

There are also techniques based on time constraints\cite{Suh2011}. These techniques convert the event-based data into tuples with quantitative variables. For example, by sliding a temporal window over the data, we can convert the occurrence of events to numeric values, and thus apply other kind of techniques such as association or regression algorithms. This might come in specially handy when taking into consideration additional variables such as the aforementioned measurements. In this case, we may disregard sequential information and focus only in frequency information, finding out how these frequencies relate to variables such as temperature or other environment data. There are four time constraints: sliding window, minimum time gap, maximum time gap and duration. A large amount of existing algorithms~\cite{Wu2010} have been implemented taking advantage of them, and can be highly useful for our project.