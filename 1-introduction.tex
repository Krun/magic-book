\chapter{Introduction}
\begin{chapterintro}
 This chapter provides an introduction to the problem which will be aproached in this project. It provides an overview of the benefits of predictive operations in systems maintenance and how data mining techniques can be used for this purpose. Furthermore, a deeper description of the project and its environment is also given.
\end{chapterintro}
\section{Data Mining and Failure Prediction}
\label{sec:about_data_mining}Data mining is the process that results in the discovery of patterns -often unknown or unexpected- in large data sets. It gathers aspects of several fields of research, such as artificial intelligence, machine learning, statistics, and database systems. The overall goal of the data mining is to extract knowledge from an existing data set and being able to extrapolate general relations which can be lately used for prediction of patterns or future acquisitions of data. Data mining is specially interesting when it allows us to predict future events based on easy measurements that can be done over the time.

In almost every field of activity, the ability to predict events in any aspect of the environment can give a significant advantage against possible competitors, or even against possible casualties which can affect or threat the quality of the activity. It is of special interest the ability to predict failures in the production systems, as they usually entail reduction of the efficiency of the activity. Thus, preventive maintenance has become a very important activity in every large business.

Data mining can offer a simple and efficient way to perform automatised preventive maintenance. The failures can only be measured and registered once they happen, not serving for the purposes of their own prevention but probably for the prediction of future related events. Additionally, there is usually a vast amount of other data which can be easily measured and monitorised from the maintained systems, such as temperature, CPU load or network activity, for example. Data mining techniques can find relations between these easily-measurable indicators and failure happening chances, allowing the implementation of alarms or even automated procedures once they reach certain levels which indicate the imminent occurrence of a failure.

In this project, we will apply data mining techniques to a specific scenario: a maintenance system of a railway network. We will focus on preventive maintenance aids, as mentioned before, using data provided by Thales, the managing company of said maintenance system. For our project, we count on a vast amount of data for the last three years, comprising both failure logs and indicator values in several maintenance stations spread throughout Spain. This provides a perfect scenario for real data mining applications, as well as a useful output for successful prediction techniques.

\vfill

\section{Project description}
The  main goal of this project is to provide a functional prediction system for failures in the given environment: a maintenance system of a railway network.

Failure prediction is the main objective for proactive maintenance in any existing system. Developing new ways of predict when errors and failures are most likely to happen can help to prepare in advance maintenance tasks, as well as to eliminate or minimize the outage time or any other inconveniences caused by these errors.

The work described in this document corresponds to the \emph{Trainmining} project developed with Thales Spain, a company which is responsible of maintenance systems for the main railway lines in Spain. Thales has gathered large amounts of event logs throughout the last years, which will be used in this project as a source to extract information which we can later use to make predictions.